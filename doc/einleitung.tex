\section{Einleitung}
%%%%%%%%%%%%%%%%%%%%%%%%%TODO Umschreiben
\begin{itemize}
 \item Ein Simpler Rechner mit den folgenden Operationen:
	\begin{itemize}
	 \item Addition
	 \item Subtraktion
	 \item Multiplikation
	 \item Division
	\end{itemize}
 \item I/O
 	\begin{itemize}
 	 \item PS/2 Keyboard als Eingabemodul
 	 \item VGA Monitor als Ausgabemodul
 	 \item RS232 Upload der 50 zuletzt eingegebenen Berechnungen
 	\end{itemize}
 \item Arithmetische Ausdruecke\\
	Es werden folgende arithmetischen Eingaben unterstützt und nach der Konvention
		der Operatorrangfolge abgearbeitet:
 	\begin{itemize}
 	 \item Digits:     0...9
 	 \item Operatoren: +, -, *, /
 	\end{itemize}
 \item Funktionelle Ausdruecke\\
 	Des Weiteres sollen folgende Funktionalen Ausdruecke unterstuetzt werden\\
 	 Backspace, Enter, Space
 \item Fehlerbehandlung
	Folgende Eingaben werden zu Abbruch bzw. zur Fehlerausgabe führen. Daraus folgt die Abspeicherung 
		eines Error Codes.
	\begin{itemize}
 	 \item Division durch 0
	 \item Ungültige Positionierung von Operanden (Anfang, Ende, zwei in Serie)
	 \item Overflows von Zahlen und Berechnungen
	\end{itemize}

\end{itemize}