\section{Einleitung}

Dieses Dokument beschreibt die Spezifikationen eines einfachen Rechners der die vier Grundrechnungsarten unterstützt.
Welcher im Zuge der LU Hardware Modellierung zu realisieren ist.\\
Eingaben werden über die Tastatur gemacht. Erlaubt sind die Zahlen '0'-'9', die vier Operationszeichen auf dem Ziffernblock, 
die Leer-, Backspace- und Entertaste des normalen Blocks.\\
Die Ausgabe erfolgt über den VGA Port des development Boards und somit auf einen Monitor. Jede Rechnung darf bis zu 70 Zeichen lang sein und das Ergebnis wird nach drücken der Entertaste
in der nächsten Zeile ausgegeben. Sollte man sich bereits in der letzten Zeile befinden, werden die restlichen Zeilen um eine Zeile nach oben
verschoben.\\
Der Taschenrechner speichert die letzten 50 Rechnungen und kann diese auf Anfrage über das RS232 Interface verschicken.

Implementiert ist auch eine Fehlerbehandlung um keine falschen Ergebnisse zu liefern. Abgedeckt ist die Division durch Null, Overflows von
Zahlen- und Ergebnissen sowie die ungültige Positionierung von Operanden.
