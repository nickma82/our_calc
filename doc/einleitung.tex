\section{Einleitung}

Dieses Dokument beschreibt die Spezifikationen eines einfachen Rechners der die vier Grundrechnungsarten unterstützt.\\ 
Eingaben werden über die Tastatur gemacht. Erlaubt sind die Zahlen '0'-'9' die vier Operationszeichen auf dem Ziffernblock und 
die Leer-, Backspace- und Entertaste vom normalen Block.\\
Die Ausgabe erfolgt über das Display. Jede Rechnung darf bis zu 70 Zeichen lang sein und das Ergebnis wird nach drücken der Entertaste 
in der nächsten Zeile ausgegeben. Sollte man bereits in der letzten Zeile befinden werden die restlichen Zeilen um eine Zeile nach oben
verschoben.\\
Der Taschenrechner speichert die letzten 50 Rechnungen und kann diese auf Anfrage über das RS232 Interface auf einen anderen PC schicken.
Implementiert ist auch eine Fehlerbehandlung um keine falschen Ergebnisse zu liefern. Abgedeckt ist die Division durch Null, Overflows von
Zahlen und Ergebnissen und die ungültige Positionierung von Operanden.
