\section{Requirements}

\begin{description}
\item[Req 1:] Eingaben werden über die Tastatur gemacht und Zeilenweise am Bildschirm ausgegeben. Beim drücken der Enter Taste
wird das Ergebnis in die nächste Zeile geschrieben. Sollte keine Zeile mehr frei sein werden die restlichen Zeilen nach 
oben verschoben und der letzte Eintrag am Bildschirm gelöscht.
\item[Req 2:] Die vier Grundrechnungsarten (+,-,*,/) müssen unterstützt werden.
\item[Req 3:] Eine gültige Zahl liegt zwischen $2^31 -1$ und $2^31$.
\item[Req 4:] Bei Divisionen wird, falls nötig, abgerundet um auf eine ganze Zahl zu kommen.
\item[Req 5:] Leerzeichen müssen eingegeben und beim Berechnen ignoriert werden.
\item[Req 6:] Die Backspace Taste löscht das letzte Zeichen und setzt den Curser zurück.
\item[Req 7:] Sollte kein Zeichen in der Rechnung stehen wird die Backspace Taste ignoriert.
\item[Req 8:] Multiplikation und Division wird vor Addition und Subtraktion berechnet. Sollten mehrere Punktrechnungen nacheinander 
ausgerechnet werden, werden diese nach der Reihenfolge ihrer Eingabe berechnet.
\item[Req 9:] Sollte die Rechnung bereits 70 Zeichen haben wird kein neues Zeichen akzeptiert.
\item[Req 10:] Der Fehler Division durch Null wird erkannt und die Fehlernachricht ``Division durch Null'' statt dem Ergebnis ausgegeben.
\item[Req 11:] Wenn zwei Zahlen, mit einem Leerzeichen getrennt, direkt nebeneinander stehen muss beim Berechnen ein Fehler erkannt und
``ungültige Syntax'' ausgegeben werden.
\item[Req 12:] Sollten zwischen zwei Zahlen zwei Operanden stehen, und der zweite Operand ist ein Minus, dann wird die zweite Zahl negativ behandelt 
und das richtige Ergebnis berechnet.
\item[Req 13:] Sollten zwischen zwei Zahlen zwei Operanden stehen, und der zweite Operand ist kein Minus, dann muss ein Fehler erkannt werden
und ``ungültige Syntax'' ausgegeben werden.
\item[Req 14:] Sollte das erste Zeichen ein Minus-Operand sein wird die erste Zahl negativ behandelt und das richtige Ergebnis ausgegeben.
\item[Req 15:] Sollte das erste Zeichen kein Minus Operand sein muss der Fehler erkannt und ``ungültige Syntax'' ausgegeben werden.
\item[Req 16:] Eine Zahl die außerhalb des Wertebereichs ist wird erkannt und beim Berechnen die Fehlernachricht ``Overflow'' ausgegeben.
\item[Req 17:] Wenn das Ergebnis mehrerer Zahlen in irgendeinem Rechenschritt außerhalb des Wertebereichs ($2^31 -1 bis 2^31$) liegt wird die
Fehlernachricht ``Overflow'' ausgegeben.
\item[Req 18:] Bei Anforderung über die serielle Schnittstelle oder beim Drücken des Buttons auf dem Entwicklerboard werden die letzten 50 Berechnungen
mit Ergebnissen über RS232 an den PC gesendet.

\end{description}